%
\let\textcircled=\pgftextcircled
\chapter{Introduction}
\label{chap:intro}

\section{Thesis Structure and Outline}

In the work comprising this thesis, we narrow our focus to active matter on a mesoscopic scale, which refers to a lengthscale ranging from approximately 10 nanometers to 10 micrometers. Organic active systems on this lengthscale, such as bacteria, tend to exist in highly dense, crowded and irregular environments. Given the persistent motion of active bodies, qualities such as these in the surrounding environment will have a large impact on the dynamics of these systems. Therefore, in this work we study the dynamics of active systems at high density as well as the dynamics in a representative complex three-dimensional environment. 

In this introductory chapter we review and detail the two aspects pertinent to this work: active matter, and structures of mesoscale confinement. Beginning with active matter, we introduce the physics governing particle motion, and then go onto to describe the unique phenomena arising from their interaction and how understanding of this behaviour derives from both theory, simulation and experiment. Following this, we introduce the background and relevant science of our method of choice for the creation of a mesoscale complex environment to host active particles: a colloidal gel. 