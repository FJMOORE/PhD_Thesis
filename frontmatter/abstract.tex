%
% File: abstract.tex
% Author: V?ctor Bre?a-Medina
% Description: Contains the text for thesis abstract
%
% UoB guidelines:
%
% Each copy must include an abstract or summary of the dissertation in not
% more than 300 words, on one side of A4, which should be single-spaced in a
% font size in the range 10 to 12. If the dissertation is in a language other
% than English, an abstract in that language and an abstract in English must
% be included.

\chapter*{Abstract}
\begin{SingleSpace}

\initial{A}ctive matter systems are comprised of bodies that consume energy from the environment for self-propulsion and as a result are driven out-of-equilibrium. These systems encompass a range of lengthscales and describe both living and artificial systems. Moreover, the motion and collective interactions of these active bodies lead to an ensemble of rich and complex behaviours not seen in equilibrium systems. On mesoscopic lengthscales (nm to $\mu$m), the prominent active systems are bacterial swimmers and active colloidal particles.
 However, as a result of the experimental challenge involved, there have been relatively few studies of mesoscale active matter systems in 3D. In this thesis, we aim to address two poorly understood areas of active systems in 3D: the crystallisation of active Brownian spheres, and the behaviour of active matter in 3D non-trivial confinement. 

Regarding crystallisation, we perform molecular dynamics simulations to explore nucleation and polymorph selection in crystals of mono-disperse active spheres. We find two activity dependant crystallisation regimes: spinodal, and nucleation and growth. In the former we observe an increase in the crystal growth rate due to the accelerated dynamics in the melt, and in the latter we find that activity inhibits nucleation. Remarkably, activity also has a strong effect on polymorph selection. While the passive system crystallises in an equimolar mixture of FCC and HCP, active particles progressively favour the FCC phase at higher activity.

The transport of mesoscale active matter occurs naturally in complex environments that are often crowded, random and irregular. We explore structural and dynamical properties of 3D active particles in random environments with a multidisciplinary approach. First we conduct molecular dynamics simulations of active Brownian spheres in two random environments. We find that the confinement increases the heterogeneity of the dynamics, with new populations of absorbed and localised particles appearing close to the obstacles. This heterogeneity has a profound impact on the motility induced phase separation  exhibited by the particles at high activity, ranging from growth in random disorder to a bicontinuous phase separation in porous environments.

Finally we approach this same problem through experiment and present the first 3D active colloidal system in non-trivial confinement. The system is comprised of active Janus particles driven via an external field, propelling through the porous network surrounding a colloidal gel. 
 This system is imaged in real-time with confocal microscopy and the particle dynamics are extracted with particle tracking methods. 



\end{SingleSpace}
\clearpage